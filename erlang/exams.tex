\subsection{2020-02-07}
We want to create a simplified implementation of the “Reduce” part of the MapReduce paradigm. To this
end, define a process “reduce\_manager” that keeps track of a pool of reducers. When it is created, it
stores a user-defined associative binary function ReduceF. It receives messages of the form {reduce,
Key, Value}, and forwards them to a different “reducer” process for each key, which is created lazily
(i.e. only when needed). Each reducer serves requests for a unique key.
Reducers keep into an accumulator variable the result of the application of ReduceF to the values they
receive. When they receive a new value, they apply ReduceF to the accumulator and the new value,
updating the former. When the reduce\_manager receives the message print\_results, it makes all its
reducers print their key and incremental result.\\
Solution:
\begin{lstlisting}
	start_reduce_mgr(ReduceF) ->
		spawn(?MODULE, reduce_mgr, [ReduceF, #{}]).
	
	reduce_mgr(ReduceF, Reducers) ->
		receive
			print_results ->
				lists:foreach(fun ({_, RPid}) -> RPid ! print_results end, maps:to_list(Reducers));
			{reduce, Key, Value} ->
				case Reducers of
					#{Key := RPid} ->
						RPid ! {Key, Value},
						reduce_mgr(ReduceF, Reducers);
					_ ->
						NewReducer = spawn(?MODULE, reducer, [ReduceF, Key, Value]),
						reduce_mgr(ReduceF, Reducers#{Key => NewReducer})
			end
		end.
	
	reducer(ReduceF, Key, Result) ->
		receive
			print_results ->
				io:format("~s: ~w~n", [Key, Result]);
			{Key, Value} ->
				reducer(ReduceF, Key, ReduceF(Result, Value))
		end.
\end{lstlisting}





\subsection{2020-01-15}
We want to implement something like Python's \texttt{range} in Erlang, using processes. \\
E.g. 
\begin{lstlisting}
    R = range(1,5,1)    % starting value, end value, step
    next(R)             % is 1
    next(R)             % is 2
    . . .
    next(R)             % is 5
    next(R)             % is the atom stop_iteration
\end{lstlisting}
Define \texttt{range} and \texttt{next}, where \texttt{range} creates a process that manages the iteration, and \texttt{next} a function that
talks with it, asking the current value. \\
Solution:
\begin{lstlisting}
    ranger(Current, Stop, Inc) ->
        receive
            {Pid, next} ->
                if
                    Current =:= Stop -> Pid ! stop_iteration;
                    true -> Pid ! Current,
                            ranger(Current + Inc, Stop, Inc)
                end 
        end.
   range(Start, Stop, Inc) ->
        spawn(?MODULE, ranger, [Start, Stop, Inc]).
   next(Pid) ->
        Pid ! {self(), next},
        receive
            V -> V
        end.
\end{lstlisting}




\subsection{2019-09-03}
\begin{itemize}
    \item 1) Define a split function, which takes a list and a number n and returns a pair of lists, where the first one is  the prefix of the given list, and the second one is the suffix of the list of length n. \\ E.g. split([1,2,3,4,5], 2) is {[1,2,3],[4,5]}
    \item 2) Using split of 1), define a splitmap function which takes a function f, a list L, and a value n, and splits L with parameter n, then launches two process to map f on each one of the two lists resulting from the split. The function splitmap must return a pair with the two mapped lists.
\end{itemize}
\begin{lstlisting}
    helper(E, {0, L}) ->
        {-1, [[E]|L]};
    helper(E, {V, [X|Xs]}) ->
        {V-1, [[E|X]|Xs]}.

    split(L, N) ->
        {_, R} = lists:foldr(fun helper/2, {N, [[]]}, L),
        R.
        
    mapper(F, List, Who) ->
        Who ! {self(), lists:map(F, List)}.

    splitmap(F, L, N) ->
        [L1, L2] = split(L, N),
        P1 = spawn(?MODULE, mapper, [F, L1, self()]),
        P2 = spawn(?MODULE, mapper, [F, L2, self()]),
        receive
            {P1, V1} ->
                receive {P2, V2} ->
                        {V1, V2}
                end
        end.
\end{lstlisting}









\subsection{2019-07-24}
Define a master process which takes a list of nullary (or 0-arity) functions, and starts a worker process for
each of them. The master must monitor all the workers and, if one fails for some reason, must re-start it to
run the same code as before. The master ends when all the workers are done. \\
Note: for simplicity, you can use the library function spawn\_link/1, which takes a lambda function, and spawns and links a process running it. \\

\begin{lstlisting}
    listlink([], Pids) -> Pids;
    listlink([F|Fs], Pids) ->
        Pid = spawn_link(F),
        listlink(Fs, Pids#{Pid => F}).

    master(Functions) ->
        process_flag(trap_exit, true),
        Workers = listlink(Functions, #{}),
        master_loop(Workers, length(Functions)).

    master_loop(Workers, Count) ->
        receive
            {'EXIT', Child, normal} ->
                if
                    Count =:= 1 -> ok;
                    true -> master_loop(Workers, Count-1)
                end;
            {'EXIT', Child, _} ->
                #{Child := Fun} = Workers,
                Pid = spawn_link(Fun),
                master_loop(Workers#{Pid => Fun}, Count)
        end.
\end{lstlisting}

\subsection{2019-06-28}
Being bu and the two cf spawn-linked, we need to terminate the main process with exit to kill them. If we do not do
so, we get an unending sequence of "no data". Hence, we could change "{P1, done} -$>$ ok" into "{P1, done} -$>$ exit(done)".
The output of the system is correct, because it is possible that some elements remain in the buffer, being the buffer managed as a
stack.

\subsection{2019-02-08}
lists:nth(V,Buf) (get element V of array B) V=1,2,3,..

\subsection{2019-01-16}
Define a process P, having a local behavior (a function), that answer to three commands: \\
\textbf{- load} is used to load a new function f on P: the previous behavior is composed with f; \\
\textbf{- run} is used to send some data D to P: P returns its behavior applied to D; \\
\textbf{- stop} is used to stop P. \\
For security reasons, the process must only work with messages coming from its creator: other messages must be discarded.
\begin{lstlisting}
    cam(Beh, Who) ->
    receive
        {run, Who, What} ->
            Who ! Beh(What),
            cam(Beh, Who);
        {load, Who, Code} ->
            cam(fun (X) -> Code(Beh(X)) end, Who);
        {stop, Who} ->
            ok;
        _ -> cam(Beh, Who)
    end.
\end{lstlisting}

\subsection{2018-09-05}
Define a function create\_pipe, which takes a list of names and creates a process of each element of the list,
each process registered as its name in the list; e.g. with [one, two], it creates two processes called 'one' and
'two'. The processes are "connected" (like in a list, there is the concept of "next process") from the last to the first,
e.g. with [one, two, three], the process structure is the following: \\
\texttt{three -> two -> one -> self}, \\
this means that the next process of 'three' is 'two', and so on; self is the process that called create\_pipe.\\
Each process is a simple repeater, showing on the screen its name and the received message, then sends it to
the next process.\\
Each process ends after receiving the 'kill' message, unregistering itself.\\
Solution:
\begin{lstlisting}
    repeater(Next, Name) ->
        receive
            kill ->
                unregister(Name),
                Next ! kill;
            V ->
                io:format("~p got ~p~n", [Name, V]),
                Next ! V,
                repeater(Next, Name)
        end.

    create_pipe([], End) -> End;
    create_pipe([X|Xs], Next) ->
        P = spawn(?MODULE, repeater, [Next, X]),
        register(X, P),
        create_pipe(Xs, X).
\end{lstlisting}

\subsection{2018-02-05}
The fixed-point of a function f and a starting value x is the value $v = f^{k}(x)$, with $k > 0$, such that $f^{k}(x) = f^{k+1}(x)$. We want to implement a fixed-point code using two communicating actors:
\begin{itemize}
    \item 1) Define the function for an \texttt{applier} actor, which has a state S, holding a value, and receives a function f from other actors: if S = f(S), it sends back the result S and ends it computation; otherwise sends back a message to state that the condition S = f(S) has not been reached.
    \item 2) Define a function called fix, which takes as input a function and a starting value, and creates and uses an applier actor to implement the fixed-point.
\end{itemize}
Solution:
\begin{lstlisting}
    applier(State) ->
        receive
            {Sender, F} -> NewState = F(State),
                if
                    NewState =:= State -> Sender!{self(), State};
                    true -> Sender!{self(), no}, applier(NewState)
                end
        end.
        loop(P, F) ->
            P!{self(), F},
            receive
                {P, V} -> if
                    V =:= no -> loop(P, F);
                    true -> V
                end
            end.
        fix(F, V) ->
        A = spawn(?MODULE, applier, [V]),
        loop(A, F).
\end{lstlisting}