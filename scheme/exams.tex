\subsection{2020-07-17}
Define the verbose construct for folding illustrated by the following example.
This is a fold-right ($->$) with initial value 1 on the list (1 2 3 4 5 6),
and the fold function is given in the ``exec'' part.
Of course, $<-$ is used to select fold-left instead of right.\\
Solution:
\begin{lstlisting}
	(define-syntax cobol-fold
		(syntax-rules (direction -> <- data using from exec)
			((_ direction -> from i data d ... (exec e ... ) using x y)
				(foldr (lambda (x y) e ...) i '(d ...)))
			((_ direction <- from i data d ... (exec e ... ) using x y)
				(foldl (lambda (x y) e ...) i '(d ...)))))
	
	(cobol-fold direction -> from 1 data 1 2 3 4 5 6
			(exec
				(displayln y)
				(+ x y))
				using x y)
\end{lstlisting}





\subsection{2020-06-29}
Define the construct define-with-types, that is used to define a procedure with type constraints, both for the parameters and for the return value.
The type constraints are the corresponding type predicates, e.g. number? to check if a value is a number.
If the type constraints are violated, an error should be issued.\\
Solution:
\begin{lstlisting}
	(define-syntax define-with-types
		(syntax-rules (:)
			((_ (f : tf (x1 : t1) ...) e1 ...)
				(define (f x1 ...)
					(if (and (t1 x1) ...)
						(let ((res (begin
								e1 ...)))
							(if (tf res)
								res
								(error "bad return type")))
						(error "bad input types"))))))
	
	(define-with-types (add-to-char : integer? (x : integer?) (y : char?))
		(+ x (char->integer y)))
	
	(add-to-char 3 #\A)
\end{lstlisting}





\subsection{2020-02-07}
Implement this new construct: (each-until var in list until pred : body), where keywords are written in
boldface. It works like a for-each with variable var, but it can end before finishing all the elements of list
when the predicate pred on var becomes true.\\
Solution:
\begin{lstlisting}
	(define-syntax each-until
		(syntax-rules (in until :)
			((_ x in L until pred : body ...)
				(call/cc
					(lambda (exit)
						(let ((pred-fun (lambda (x) pred)))
							(let loop ((xs L))
								(if (null? xs)
									(exit)
									(let ((x (car xs)))
										(if (pred-fun x)
											(exit)
											(begin
												body ...
												(loop (cdr xs)))))))))))))
\end{lstlisting}





\subsection{2020-01-15}
Consider the Foldable and Applicative type classes in Haskell. We want to implement something
analogous in Scheme for vectors. Note: you can use the following library functions in your code: vectormap, vector-append.
\begin{itemize}
    \item Define vector-foldl and vector-foldr.
    \item Define vector-pure and vector-$<*>$.
\end{itemize}
Solution:
\begin{lstlisting}
    (define (vector-foldr f i v)
        (let loop ((cur (- (vector-length v) 1))
                    (out i))
            (if (< cur 0)
                out
                (loop (- cur 1) (f (vector-ref v cur) out)))))
    (define (vector-foldl f i v)
        (let loop ((cur 0)
                    (out i))
            (if (>= cur (vector-length v))
                out
                (loop (+ cur 1) (f (vector-ref v cur) out)))))
    (define (vector-concat-map f v)
        (vector-foldr vector-append #() (vector-map f v)))
    (define vector-pure vector)
    (define (vector-<*> fs xs)
        (vector-concat-map (lambda (f) (vector-map f xs)) fs))
\end{lstlisting}







\subsection{2019-09-03}
Consider the following code:
\begin{lstlisting}
    (define (a-function lst sep)
        (foldl (lambda (el next)
                (if (eq? el sep)
                    (cons '() next)
                    (cons (cons el (car next))
                        (cdr next))))
            (list '()) lst))
\end{lstlisting}
\begin{itemize}
    \item 1) Describe what this function does; what is the result of the following call? \\
            \texttt{(a-function '(1 2 nop 3 4 nop 5 6 7 nop nop 9 9 9) 'nop)}
    \item 2) Modify a-function so that in the example call the symbols nop are not discarded from the resulting list,
    which must also be reversed (of course, without using reverse).
\end{itemize}
Solution:
\begin{lstlisting}
    a-function returns a list of lists, where each list is taken backwards, and sep is used for a separator. The resulting list is: ((9 9 9) () (7 6 5) (4 3) (2 1))
    The modified function is:
    (define (another-function lst sep)
        (foldr (lambda (el next)
                (if (eq? el sep)
                    (cons (list el) next)
                    (cons (cons el (car next))
                        (cdr next))))
            (list '()) lst))
\end{lstlisting}




\subsection{2019-07-24}
Write a functional, tail recursive implementation of a procedure that takes a list of numbers L and two values
x and y, and returns three lists: one containing all the elements that are less than both x and y, the second one
containing all the elements in the range [x,y], the third one with all the elements bigger than both x and y. It
is not possible to use the named let construct in the implementation. \\
Solution:
\begin{lstlisting}
    (define (3-part L v1 v2)
        (define (3-p L v1 v2 r1 r2 r3)
            (if (null? L)
                (list r1 r2 r3)
                (let    ((x (car L))
                        (xs (cdr L)))
                    (cond
                        ((and (< x v1)(< x v2))
                            (3-p xs v1 v2 (cons x r1) r2 r3))
                        ((and (>= x v1)(<= x v2))
                            (3-p xs v1 v2 r1 (cons x r2) r3))
                        ((and (> x v1)(> x v2))
                            (3-p xs v1 v2 r1 r2 (cons x r3)))))))
        (3-p L v1 v2 '() '() '()))
\end{lstlisting}





\subsection{2019-06-28}
Consider this data definition in Haskell: data Tree a = Leaf a | Branch (Tree a) a (Tree a) \\
Define an OO analogous of this data structure in Scheme using the technique of "closure as classes" as seen in class, defining the map and print methods, so that: \\
\texttt{(define t1 (Branch (Branch (Leaf 1) -1 (Leaf 2)) -2 (Leaf 3)))} \\
\texttt{((t1 'map (lambda (x) (+ x 1))) 'print)} \\
should display: \texttt{(Branch (Branch (Leaf 2) 0 (Leaf 3)) -1 (Leaf 4))} \\
Solution:
\begin{lstlisting}
    (define (Branch t1 x t2)
        (define (print)
            (display "(Branch ")
            (t1 'print)
            (display " ")
            (display x)
            (display " ")
            (t2 'print)
            (display ")"))
        (define (map f)
            (Branch (t1 'map f) (f x) (t2 'map f)))
        (lambda (message . args)
            (apply
                (case message
                    ((print) print)
                    ((map) map)
                    (else (error "Unknown")))
                args)))

    (define (Leaf x)
        (define (print)
            (display "(Leaf ")
            (display x)
            (display ")"))
        (define (map f)
            (Leaf (f x)))
        (lambda (message . args)
            (apply
                (case message
                    ((print) print)
                    ((map) map)
                    (else (error "Unknown")))
                args)))
\end{lstlisting}

\subsection{2019-02-08}
Define a \underline{pure} function \texttt{multi-merge} with a variable number of arguments (all of them must be ordered lists of
numbers), that returns an ordered list of all the elements passed. \underline{It is forbidden to use external sort functions.} \\
E.g. when called like: \\
\texttt{(multi-merge '(1 2 3 4 8) '(-1 5 6 7) '(0 3 8) '(9 10 12))} \\
it returns: \texttt{'(-1 0 1 2 3 3 4 5 6 7 8 8 9 10 12)} \\
Solution:
\begin{lstlisting}
    (define (merge a1 a2)
        (cond ((null? a1) a2)
            ((null? a2) a1)
            (else   (let ((x (car a1))
                          (y (car a2)))
                        (if (< x y)
                            (cons x (merge (cdr a1) a2))
                            (cons y (merge a1 (cdr a2))))))))
    (define (multi-merge . data)
        (foldl merge '() data))
\end{lstlisting}

\subsection{2019-01-16}
Define a pure function f with a variable number of arguments, that, when called like \texttt{(f x1 x2 .. xn)}, returns:
\texttt{(xn (xn-1 ( .. (x1 (xn xn-1 .. x1))..)}. Function \texttt{f} must be defined using only fold operations for loops.
\begin{lstlisting}
    (define (f . L)
        (foldl  (lambda (x y)
                    (list x y))
                (foldl cons '() L)
                L))
\end{lstlisting}