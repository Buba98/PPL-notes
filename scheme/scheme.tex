

\subsection{Lambda}
Lambdas are unnamed procedures
\begin{lstlisting}[language=Lisp]
    (lambda (x y) ; this is a comment
        (+ (* x x) (* y y)))
\end{lstlisting}
Example usage:
\begin{lstlisting}[language=Lisp]
    ((lambda (x y) (+ (* x x) (* y y))) 2 3) ; => 23
\end{lstlisting}

\subsection{Let}
Scope is \textbf{static}
let example:
\begin{lstlisting}[language=Lisp]
    (let ((x 2)
          (y 3))
        ;here go the code using x,y
    )
\end{lstlisting}
let binds variables "in parallel":
\begin{lstlisting}[language=Lisp]
    (let ((x 1)
          (y 2))
        (let ((x y)
            (y x))
            ;y = 1, x = 2 
        )   
    )
\end{lstlisting}

\subsection{Quoting}
Syntax form to prevent evaluation
\begin{lstlisting}[language=Lisp]
    (quote <expr>)
    ;equivalent to
    '<expr>
\end{lstlisting}
<expr> is left unevaluated
\begin{lstlisting}[language=Lisp]
    (quote (1 2 3)) ;is a list
    (quote (+ 1 2)) ;is another list
\end{lstlisting}
\textit{quasiquote} (‘) and \textit{unquote} (,) are used for partial evaluation
\begin{lstlisting}[language=Lisp]
    '(1 2 3) ;is a list (1 2 3)
    `(1 ,(+ 1 1) 2) ;is another list (1 2 3)
\end{lstlisting}

\subsection{Eval}
\begin{lstlisting}[language=Lisp]
    eval '(+ 1 2 3) ;is 6
\end{lstlisting}

\subsection{Begin}
Use \textbf{begin} to write block of procedural code:
\begin{lstlisting}[language=Lisp]
    (begin
        (op-1 ...)
        (op-2 ...)
        ...
        (op-n ...)
    )
\end{lstlisting}

Every op-i is evaluated in order and the value of the block is the value obtained by the last expression

\subsection{Definitions}
To create top-level bindings there is \textbf{define}:
\begin{lstlisting}[language=Lisp]
    (define <name> <what>)
    ;example
    (define x 12)
    (define y #(1 2 3))
\end{lstlisting}
Defining procedures:
\begin{lstlisting}[language=Lisp]
    (define cube (lambda (x) (* x x x)))
    (cube 3) ; = 27
\end{lstlisting}
Short notation to define procedures:
\begin{lstlisting}[language=Lisp]
    (define (cube x) (* x x x))
    (cube 3) ; = 27
\end{lstlisting}

\subsection{Assignment}:
Use \textbf{set!} for assignement:
\begin{lstlisting}[language=Lisp]
    (begin
        (define x 23)
        (set! x 42)
        x
    )  ; = 42
\end{lstlisting}

\subsection{List}
Lists are memorized as concatenated pairs, a pair (written (x . y), also called a cons node) consists of two parts:
\begin{itemize}
    \item \textbf{car} (i.e. x) aka Content of the Address Register (take the first element of a list)
    \item \textbf{cdr} (i.e. y) aka Content of the Data Register (take the list without the first element)
\end{itemize}
A list (1 2 3) is stored like this (1 . (2 . (3 . ()))) \
() is the empty list \
The two procedures \textbf{car} and \textbf{cdr} are used as accessors \
\textbf{member} is used to check if a list contains a value:
\begin{lstlisting}[language=Lisp]
    (member 2 '(1 2 3)) ; => '(2 3)
\end{lstlisting}

Define a procedure with variable number if argument
\begin{lstlisting}[language=Lisp]
    (define (x . y) y)
    (x 1 2 3)  ; = > '(1 2 3)
\end{lstlisting}
\textbf{apply} can be used to apply procedure to list of elements:
\begin{lstlisting}[language=Lisp]
    (apply + '(1 2 3 4)) ; = 10
\end{lstlisting}
To build pair we can use \textbf{cons}
\begin{lstlisting}[language=Lisp]
    (cons 1 2) ;    = (1 . 2)
    (cons 1 '(2)) ; = (1 2)
\end{lstlisting}
Example, find minimum of a list:
\begin{lstlisting}[language=Lisp]
    (define (minimum L)
        (let ((x (car L))
              (xs (cdr L)))
            (if (null ? xs) ; is xs = ()?
                x           ; then return x
            (minimum        ; else : recursive call
                (cons
                    (if (< x (car xs))
                        x
                        (car xs))
                    (cdr xs))))))
    (minimum '(11 -3 2 3 8 -15 0)) ; = > -15
\end{lstlisting}
Variant with variable number of argument:
\begin{lstlisting}[language=Lisp]
    (define (minimum x . rest)
        (if (null ? rest)   ; is rest = ()?
            x               ; then return x
            (apply minimum  ; else : recursive call
                (cons
                    (if (< x (car rest))
                        x
                        (car xs)
                    )
                    (cdr xs)
                )
            )
        )
    )
    (minimum 11 -3 2 3 8 -15 0) ; = > -15
\end{lstlisting}

\subsection{Loop}
Example:
\begin{lstlisting}[language=Lisp]
    (let label ((x 0))
        (when (< x 10)
            (display x)
            (newline)
            (label (+ x 1))
        )
    )
\end{lstlisting}
We can use as many variables as we like, value obtained by the last expression.

\subsection{Tail recursion}
Example \textbf{not tail recursive}:
\begin{lstlisting}[language=Lisp]
    (define (factorial n)
        (if (= n 0)
            1
            (* n factorial (- n 1))
        )
    )
\end{lstlisting}
Example \textbf{tail recursive}:
\begin{lstlisting}[language=Lisp]
    (define (fact x)
        (define (fact-tail x accum)  ;local proc
            (if (= x 0)
                accum
                (fact-tail (- x 1) (* x accum))
            )
        )
        (fact-tail x 1)
    )
\end{lstlisting}
To avoid stack consumption it can be optimize (indeed, the previous tail call is translated in the following low-level code)
\begin{lstlisting}[language=Lisp]
    (define (fact-low-level n)
        (define x n)
        (define accum 1)
        (let loop () ; see this as the "loop" label
            (if (= x 0)
                accum
                (begin
                    (set! accum (* x accum))
                    (set! x (- x 1))
                    (loop)  ; jump to "loop"
                )
            )
        )  
    )
\end{lstlisting}
A more idiomatic way of writing it is the following:
\begin{lstlisting}[language=Lisp]
    (define (fact-low-level n)
        (let loop ((x n)
              (accum 1))
            (if (= x 0)
                accum
                (loop (- x 1)(* accum x))
            )
        )  
    )
\end{lstlisting}
But note that this looks like a tail call. . .
(In reality, the named let is translated into a local recursive function. If tail
recursive, when compiled it becomes a simple jump.)

\subsection{for-each}
\begin{lstlisting}[language=Lisp]
    (for-each (lambda (x)
                (display x)(newline))
            '(this is it)
    )
\end{lstlisting}
For vector:
\begin{lstlisting}[language=Lisp]
    (vector-for-each (lambda (x)
                        (display x)(newline))
                    #(this is it)
    )
\end{lstlisting}
Here is the definition:
\begin{lstlisting}[language=Lisp]
    (define (vector-for-each body vect)
        (let ((max (- (vector-leght vect) 1)))
            (let loop ((i 0))
                (body (vector-ref vect i)) ; vect[i] in C
                (when (< i max)
                    (loop (+ x 1))
                )
            )
        )
    )
\end{lstlisting}

\subsection{Equivalence}
\begin{itemize}
    \item A predicate is a procedure that returns a Boolean. Its name usually ends with ? (e.g. null?)
    \item = is used only for numbers
    \item there are:
    \begin{itemize}
        \item \textbf{eq?} tests if two objects are the same (good for symbols) \textit{(eq? ’casa ’casa), but not (eq? "casa" (string-append "ca" "sa")), (eq? 2 2) is unspecified}
        \item \textbf{eqv?} like eq?, but checks also numbers
        \item \textbf{equal?} predicate is true iff the (possibly infinite) unfoldings of its arguments into regular trees are equal as ordered trees. \textit{(equal? (make-vector 5 'a)(make-vector 5 'a)) is true}
    \end{itemize}
\end{itemize}

\subsection{Case and Cond}
\begin{itemize}
    \item case:
        \begin{lstlisting}[language=Lisp]
            (case (car '(c d))
                ((a e i o u)    'vowel)
                ((w y)          'semivowel)
                (else           'constant) 
            )   ; => constant
        \end{lstlisting}
    \item cond:
        \begin{lstlisting}[language=Lisp]
            (cond   ((> 3 3) 'greater)
                    ((< 3 3) 'less)
                    (else    'equal)
            )   ; => equal
        \end{lstlisting}
\end{itemize}

\subsection{Storage model and mutability}
\begin{itemize}
    \item Variables and object implicitly refer to locations or sequence of locations in memory (usually the heap)
    \item Scheme is garbage collected: every object has unlimited extent – memory used by objects that are no longer reachable is reclaimed by the GC
    \item Constants reside in read-only memory (i.e. regions of the heap explicitly marked to prevent modifications), therefore literal constants (e.g. the vector \#(1 2 3)) are immutable
    \item If you need e.g. a mutable vector, use the constructor (vector 1 2 3)
    \item Mutation, when possible, is achieved through "bang procedures", e.g. (vector-set! v 0 "moose")
\end{itemize}
Vector are mutable only if created with standard constructor
\begin{lstlisting}[language=Lisp]
    (define (f)
        (let ((x (vector 1 2 3)))
            x
    )   )
    
    (define v (f))
    (vector-set! v 0 10)
    (display v)(newline) ; => #(10 2 3)
\end{lstlisting}
Vector are immutable if created with \#(x y z)
\begin{lstlisting}[language=Lisp]
    (define (g)
        (let ((x #(1 2 3)))
            x
    )   )

    (display (g))(newline) ; = > #(1 2 3)
    (vector-set! (g) 0 10) ; = > error !
\end{lstlisting}
List:
\begin{itemize}
    \item In Racket, lists are immutable (so no set-car!, set-cdr!) - this is different from most Scheme implementations (but it is getting more common)
    \item There is also a mutable pair datatype, with mcons, set-mcar!, set-mcdr!
\end{itemize}

\subsection{Evaluation Strategy}
Evaluation strategy: call by object sharing (like in Java): objects are allocated on the heap and references to them are passed by value.
\begin{lstlisting}[language=Lisp]
    (define (test-setting-local d)
        (set! d "Local") ; setting the local d
        (display d)(newline))

    (define ob "Global")
    (test-setting-local ob) ; = > Local
    (display ob)            ; = > Global
\end{lstlisting}
\begin{itemize}
    \item It is also often called call by value, because objects are evaluated before the call, and such values are copied into the activation record
    \item The copied value is not the object itself, which remains in the heap, but a reference to the object
    \item This means that, if the object is mutable, the procedure may exhibit side effects on it
\end{itemize}
\begin{lstlisting}[language=Lisp]
    (define (set-my-mutable d)
        (vector-set! d 1 "done")
        (display d))

    (define ob1 (vector 1 2 3)) ; i.e. #(1 2 3)
    (set-my-mutable ob1)    ; = > #(1 done 3)
    (display ob1)           ; = > #(1 done 3)
\end{lstlisting}

\subsection{Struct}
It is possible to define new types, through \texttt{struct}.
The main idea is like struct in C, with some differences.
\begin{lstlisting}[language=Lisp]
    (struct being (
        name            ; name is immutable
        (age #:mutable) ; flag for mutability
    ))
\end{lstlisting}
A number of related procedures are automatically created, e.g. the constructor \texttt{being} and a predicate to check if an object is of this type: \texttt{being?} in this case \
Also accessors (and setters for mutable fields) are created \
e.g., we can define the following procedure:
\begin{lstlisting}[language=Lisp]
    (define (being-show x)
        (display (being-name x))
        (display " (")
        (display (being-age x))
        (display ") ")
    )
    (define (say-hello x)
        (if (being ? x) ; ; check if it is a being
            (begin
                (being-show x)
                (display " : my regards . ")
                (newline))
            (error " not a being " x)
        )
    )
    ; Example usage:
    (define james (being "James" 58))
    (say-hello james)   ; => James (58): my regards.
    (set-being-age! james 60) ; a setter
    (say-hello james)   ; => James (58): my regards.
    ; it is not possible to change its name
\end{lstlisting}
Structs can inherit
\begin{lstlisting}[language=Lisp]
    (struct may-being being (   ; being is the father
        (alive? #:mutable)      ; to be or not to be
    ))

    (define (kill! x)
        (if (may-being? x)
            (set-may-being-alive?! x #f)
            (error "not a may-being"))
    )

    (define (try-to-say-hello x)
        (if (and (my-being? x) (not (may-being-alive? x)))
            (begin
                (display "I hear only silence.")
                (newline)
            )
            (say-hello x))
    )
    ; Example usage
    (define john (may-being "Jhon" 77 #t))
    (say-hello jhon)    ; => John (77): my regards.
    (kill! jhon)
    (try-to-say-hello john) ; => I hear only silence.
\end{lstlisting}

\subsection{Structs vs Object-Oriented programming}
The main difference is in methods vs procedures:
\begin{itemize}
    \item procedures are external, so with inheritance we cannot redefine/override them
    \item still, a may-being behaves like a being
    \item but we had to define a new procedure (i.e. try-to-say-hello), to cover the role of say-hello for a may-being
    \item structs are called records in the standard.
\end{itemize}


\subsection{Closures}




\subsection{Some classical higher order functions}
\begin{itemize}
    \item \textbf{map}: \begin{equation} map(f,(e_1,e_2,\ldots,e_n)) = (f(e_1),f(e_2),\ldots,f(e_n)) \end{equation}
    \item \textbf{filter}: \begin{equation} filter(p(e_1,e_2,\ldots,e_n)) = (e_i|1\leq i\leq n,p(e_i)) \end{equation}
    \item \textbf{foldr}: \begin{equation} fold_{right}(\circ ,\iota,(e_1,e_2,\ldots,e_n)) = (e_1\circ(e_2\circ\ldots(e_n\circ\iota))) \end{equation}
    \item \textbf{foldl}: \begin{equation} fold_{left}(\circ ,\iota,(e_1,e_2,\ldots,e_n)) = (e_n\circ(e_{n-1}\circ\ldots(e_1\circ\iota))) \end{equation}
\end{itemize}
\todo[inline]{example tail-recursive foldr}


\subsection{Meta-programming through Macros}
\todo[inline]{Example while and let}

\subsection{Hygiene}
\begin{itemize}
    \item Scheme macros are hygienic
    \item this means that symbols used in their definitions are actually replaced with special symbols not used anywhere else in the program
    \item therefore it is impossible to have name clashes when the macro is expanded
    \item on the other hand, sometime we want name clashes, so these particular cases can be tricky (we will see an example later)
\end{itemize}

    
\subsection{Continuation}
\todo[inline]{Continuation}


\subsection{Exceptions}
\todo[inline]{Exceptions}

\subsection{Object Oriented programming}
It is possible to use closures to do some basic OO programming
\begin{lstlisting}[language=Lisp]
    (define (make-simple-object)
        (let ((my-var 0)) ; attribute
            ; methods :
            (define (my-add x)
                (set ! my-var (+ my-var x))
                my-var)
            (define (get-my-var)
                my-var)
            (define (my-display)
                (newline)
                (display "My Var is : ")
                (display my-var)
                (newline))
            (lambda (message . args)
                (apply (case message
                    ((my-add)       my-add)
                    ((my-display)   my-display)
                    ((get-my-var)   get-my-var)
                    (else (error " Unknown Method ! ")))
                args))))
\end{lstlisting}
Example usage:
\begin{lstlisting}[language=Lisp]
    (define a (make-simple-object))
    (define b (make-simple-object))
    (a 'my-add 3)       ; => 3
    (a 'my-add 4)       ; => 7
    (a 'get-my-var)     ; => 7
    (b 'get-my-var)     ; => 0
    (a 'my-display)     ; => My Var is : 7
\end{lstlisting}

\subsection{Inheritance by delegation}
\begin{lstlisting}[language=Lisp]
    (define (make-son)
        (let    ((parent    (make-simple-object))  ; inheritance
                 (name       "anobject"))          
            ; methods :
            (define (hello)
                "hi!")
            (define (my-display)
                (newline)
                (display "My name is ")
                (display name)
                (display " and")
                (parent 'my-display))
            (lambda (message . args)
                (case message
                    ((hello)        (apply hello args))
                    ; override
                    ((my-display)   (apply my-display args))
                    ; parent needed
                    (else (apply parent (cons message args)))))))
\end{lstlisting}
Example usage:
\begin{lstlisting}[language=Lisp]
    (define c (make-son))
    (c 'my-add 2)        ; => 2
    (c 'my-display)      ; => My name is an object and
                         ;    My Var is : 2
    (display (c 'hello)) ; => hi!
\end{lstlisting}

\subsection{Prototype-Based Object system}
Some hash table function:
\begin{lstlisting}[language=Lisp]
(hash-ref hash key [failure-result]) -> any ;Returns the value for key in hash. If no value is found for key, then failure-result determines the result
(hash-set! hash key v) -> void? ; Maps key to v in hash, overwriting any existing mapping for key.
\end{lstlisting}

An object is implemented with a hash table

\begin{lstlisting}[language=Lisp]
    (define new-object make-hash)
    (define clone hash-copy)
\end{lstlisting}
keys are attribute/method names. \
Just for convenience:
\begin{lstlisting}[language=Lisp]
    (define-syntax !!       ; setter
        (syntax-rules ()
            ((_ object msg new-val)
             (hash-set! object 'msg new-val))))
    (define-syntax ??       ; reader
        (syntax-rules ()
            ((_ object msg)
             (hash-ref object 'msg))))
    (define-syntax - >      ; send message
        (syntax-rules ()
            ((_ object msg arg ...)
             ((hash-ref object 'msg) object arg ...))))
\end{lstlisting}
First, we define an object and its methods
\begin{lstlisting}[language=Lisp]
    (define Pino (new-object))
    (!! Pino name "Pino")   ; slot added
    (!! Pino hello 
        (lambda (self x)
            (display (?? self name))
            (display ": hi, ")
            (display (?? x name"))
            (display "!")
            (newline)))
    (!! Pino set-name
        (lambda (self x)
            (!! self name x)))
    (!! Pino set-name-&-age
        (lambda (self n a)
            (!! self name n)
            (!! self age a)))
\end{lstlisting}
Now create object Pina as clone of Pino
\begin{lstlisting}[language=Lisp]
    (define Pina (clone Pino))
    (!! Pina name "Pina")
    ; example usage
    (-> Pino hello Pina)    ; Pino: hi, Pina!
    (-> Pino set-name "Ugo")
    (-> Pina set-name-&-age "Lucia" 25)
    (-> Pino hello Pina)    ; Ugo: hi, Lucia!
\end{lstlisting}


\subsection{Proto-OO: Inheriance}
\begin{lstlisting}[language=Lisp]
    (define (son-of parent)
        (let ((o (new-object)))
            (!! o <<parent>> parent)
            o))
    ; Basic dispatching
    (define (dispatch object msg)
        (if (eq? object 'unknown)
            (error "Unknown message" message)
            (let ((slot (hash-ref object msg 'unknown)))
                (if (eq? slot 'unknown)
                    (dispatch (hash-ref object '<<parent>> 'unknown) msg)
                    slot))))

    ; Have to modify ?? and -> for dispatching
    (define-syntax ??       ; reader
        (syntax-rules ()
            ((_ object msg)
             (dispatch object 'msg))))
    (define-syntax - >      ; send message
        (syntax-rules ()
            ((_ object msg arg ...)
             ((dispatch object 'msg) object arg ...))))
\end{lstlisting}
Example usage:
\begin{lstlisting}[language=Lisp]
    (define Glenn (son-of Pino))
    (!! Glenn name "Glenn")
    (!! Glenn age 50)
    (!! Glenn get-older
        (lambda (self)
            (!! self age (+ 1 (?? self age)))))
    
    (-> Glenn hello Pina)   ; Glenn: hi, Pina!
    (-> Glenn ciao)         ; error: Unknown message
    (-> Glenn get-older)    ; Glenn is now 51
\end{lstlisting}